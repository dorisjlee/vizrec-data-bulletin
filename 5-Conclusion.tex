%!TEX root = main.tex
\section{Concluding Remarks\label{sec:conclusion}}
\par Datasets are inherently agnostic to system and users. Visual data exploration systems is helps users bridge the gap between what they want from the data (querying) and what the data has to offer as insights(recommendations). This paper discusses how to better support these and how ----. Section~\ref{sec:precise} discusses how precise visual query systems provide informative visualizations to accelerate the process of data discovery. Section~\ref{sec:vague} discusses mixed-intiative systems that accounts for vague and complex query and allow users to provide feedback to refine their queries. Section \ref{sec:understanding} discusses the challenges and opportunities in moving from intention-driven querying to facilitating more integrated data understanding and awareness during the analysis workflow.

% Data is agnostic to the user, intention ---, by building tools---, Section \ref{sec:precise} to \ref{sec:vague} have focussed on extracting what user want from data. bridging together what user want from data, what data has to offer, supporting interactive discourse between the two. 

% \par In this paper, we advocate supporting a desiderata consisting of  3`I's in the cycle of visual data exploration:
% \begin{itemize}
% 	\item \textbf{Informative}: Section~\ref{sec:precise} discusses how precise visual query systems provide informative visualizations to accelerate the process of data discovery. 
% 	\item \textbf{Iterative}: Section~\ref{sec:vague} discusses mixed-intiative systems that accounts for vague and complex query and allow users to provide feedback to refine their 
% 	Joining the flow, query refinement, dialogue (not a one-shot query), feedback and recommendation, expressivity (how easy is it to express what to do via interactions) and diversity of actions that could be performed.
% 	\item \textbf{Integrated}: Section \ref{sec:understanding} discusses the challenges and opportunities in moving from intention-driven querying to facilitating more integrated data understanding and awareness during the analysis workflow.
% \end{itemize}

%  Either using one-size-fits-all statistics, templates, heuristics as a solution or problem only applicable to a subset of analytic tasks\cite{Vartak2015,Vartak2017}. 
