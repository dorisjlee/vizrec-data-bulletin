%!TEX root = main.tex
\section{Hypothesis Formation and the cycle of visual analysis\label{sec:hypothesis}}
\subsection{Visual Querying in the \textit{Sensemaking Framework}}
In developing \zv, we collaborated with scientists from astronomy, genetics, and material science in a year-long participatory design process~\cite{Lee2017}. In particular, we studied how various features impact analysts' ability to rapidly generate new hypotheses and insights and perform visual querying and analysis. We employed Pirolli and Card's information foraging theory to contextualize our study results~\cite{Pirolli}. Our findings not only offers design guidelines for improving the usability and adoption of next-generation PVQSs, but more importantly, points towards the need for supporting other components in the cycle of visual data exploration. %can be applied to the context of supporting the full cycle of visual data exploration. 
\subsection{The Problem of Vague and Complex Visual Queries}
%\subsection{The Need for Visual Querying Expressivity}%Supporting Vague and Complex Querying
\par When users query with a PVQS, they often translate their ambiguous, high-level questions into an plan that consists of multiple interactions to incrementally address their desired query. Participants in our \zv design study often created unexpected workflows that chained together analysis steps consisting of multiple interactions and controls. For example, geneticists in our study repeatedly explored representative trends to gain an overall sense of the typical profiles that exist in their dataset and queried mainly through drag-and-drop of these representative trends. Variations to their main workflow also include changing cluster sizes and display control settings to offer them different perspectives on the dataset. 
\par The expressiveness of PVQSs comes from the multiplicative effect of stringing together combinations of interaction sequences into a customized workflow. Designing features that diversifies potential variations expands the space of possible workflows that could be constructed during the analysis. However, even with many supported interactions, there were still vague and complex queries that could not be decomposed into a multi-step interaction workflow. In Section~\ref{sec:vague}, we highlight the challenges of PVQSs in supporting vague and complex queries and several promising directions of ongoing research in this area.
%related work in the area of vague and complex querying. More importantly pointed towards a need for vague querying. Give some examples of vague querying.
%The main workflow for the astronomers in our user study involves \textit{enriching}, either through the creation of groups or via filtering data subsets. The main workflow for material scientists involves \textit{exploiting}, since they spend the majority of their efforts performing ``close-reading'' of individual visualizations to understand the relationships between  physical variables. 
% \subsection{The Need for Recommendations for data ----- awareness}
% % intention-free 
\subsection{The Problem of Prescriptive Visual Querying}
% \subsection{Top-down and Bottom-up Querying Modalities}
% \par Our study also revealed the challenges of coming up with a query in a prescriptive manner. 
% Need a topic sentence/paragraph
\par Pirolli and Card's notional model distinguishes between information processing tasks that are \textit{top-down} (from theory to data) and \textit{bottom-up} (from data to theory)~\cite{Pirolli}. In the context of visual querying, users employ top-down approaches by starting with a hypothesis on what patterns to look for and express it through sketching or inputting an equation (Figure~\ref{fig:modalities}b,d). On the other hand, bottom-up approaches originate from the data (or equivalently, the visualization). For example, the user may drag and drop a visualization of interest in the dataset as the input query or upload a visualization from an external dataset (Figure~\ref{fig:modalities}a,c). These two modalities of querying are reminiscent of the browsing and searching behaviors on the Web, which derives from the successful application of foraging theory to Web Search~\cite{Olston2003}.
\par While the usage of each querying feature may vary from one participant to the next, our interactions with the scientists showed that \emph{bottom-up querying via drag-and-drop was more intuitive and more commonly used than top-down querying methods when the users have no desired patterns in mind}, which is commonly the case for exploratory data analysis. One of the main reason why participants did not find sketching useful was that they often do not start their analysis with a pattern in mind. Later, their intuition about what to query is derived from other visualizations that they see in the PVQS, in which case it made more sense to query using those visualizations as examples directly. Similarly, while functional fitting is a common operation in scientific data analysis, querying by equation is also unpopular, since it is challenging to formulate functional forms in an prescriptive, ad-hoc manner without seeing what the common patterns in the dataset are. 
\par A key design principle that came from this study was the need for visualization recommendations that can help analysts jump-start their exploration. For example, we found that many users made use of the recommended representative trends and outliers visualizations provided by \zv as contextual information to better understand their data or to query based on these recommended visualizations. %(e.g. after a filter is applied)  (e.g. find visualizations that are similar to the one in the largest representative clusters). 
\par As evident from the representative and outliers visualization recommendations in \zv, recommendations facilitate a smoother flow of analysis by closing the loop between the two modalities of querying and exploration, thus ensuring that user is never stuck or out of ideas at any point during the analysis. Typically, visualization recommendation systems seeks to accelerate the process of discovering interesting aspects of the data by broadening exploration. In Section~\ref{sec:understanding}, we advocate recommendation systems should not only focus on data discoverability aspect of exploration, but also contribute towards helping users become more aware of the distributions in their data and the context of their analysis.
%gain better awareness and understanding of the scope and context of their analysis and data.
%for the importance of building recommenders provides that contributes towards data awareness and understanding but also -----. One ----- it does this by going towards better data understanding.
%. an accurate understanding of the context of analysis and scope of data.  but also contribute towards ---- data understanding.
%thought of through the notion of data discoverability ----understanding and finding unseen data. 
% \subsection{Challenges Ahead}
% The goal here is to help novice submit precise queries without SQL background, easy to use interface. Our study found that VQS does more than just this, but still not enough.
% \begin{itemize}
% 	\item Precise Search Fail to understand intricacies of user need/intent, need more expressivity/flexibility for querying.
% 	\item  No perfect training workload, real-world data + task is noisy and complex. 
% 	\item towards more holistic model for insight discovery
% \end{itemize}

%We highlight two of the key findings related to this goal below.: Our participatory design findings points towards future ----- in supporting a cycle of ---.  ----advocate ---cycle of visual analysis. 
% \begin{itemize}
% 	\item Essential ingredient in facilitating intelligent vague querying and exploration.
% 	\item This is a human process (\cite{Heer2012,Pirolli})
% 	\item Iterative Hypothesis Exploration/Refinement : argue that the following properties is important to sustain this “cycle of visual analysis” 
% \end{itemize}