%!TEX root = main.tex
\section{Hypothesis Formation\label{sec:hypothesis}}
\subsection{Supporting cycle of visual analysis}
\begin{itemize}
	\item Essential ingredient in facilitating intelligent vague querying and exploration.
	\item This is a human process (\cite{Heer2012,Pirolli})
	\item Iterative Hypothesis Exploration/Refinement : argue that the following properties is important to sustain this “cycle of visual analysis” 
\end{itemize}
\subsection{Visual Querying in the framework of Data Sensemaking}
\cite{Lee2017}
Our participatory design ---situating visual querying in the framework of ------. Our \zv work ------- Bottom up and top down querying in VQS facilitates rapid insight discovery. More importantly pointed towards a need for vague querying. Give some examples of vague querying.

\subsection{Towards 3‘I’s of rapid hypothesis generation support}
Given our observations from the participatory design study, we distill several desiderata for the next generation VQSs. 
Towards 3‘I’s Interactive, Iterative, Informative (Give examples from the ZV-TVCG paper)
\stitle{Interactive flow}:
 (how natural is it to move between analysis steps, facilitate fluid analysis and not get “stuck”) : interactivity, feedback (latter is quite unexplored), and recommendation, expressivity (how easy is it to express what to do via interactions) and diversity of actions that could be performed.
\stitle{Iterative}: query refinement, dialogue (not a one-shot query)
Joining the flow: Section \ref{sec:vague} focusses on the first two items .
\stitle{Informative}: not just task-based interestingness but more explanation-based (causality, introduce distribution awareness notion in viz-sum), focussed on data understanding, which we will discuss in Section \ref{sec:understanding}