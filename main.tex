\documentclass[11pt]{article}
\usepackage{deauthor}
\usepackage{times}
\usepackage{graphicx}
\usepackage{xspace}
\usepackage{xcolor}
\usepackage[square,numbers]{natbib}

\newenvironment{denselist}{
    \begin{list}{\small{$\bullet$}}%
    {\setlength{\itemsep}{0ex} \setlength{\topsep}{0ex}
    \setlength{\parsep}{0pt} \setlength{\itemindent}{0pt}
    \setlength{\leftmargin}{1.5em}
    \setlength{\partopsep}{0pt}}}%
    {\end{list}}

\newcommand{\squishlist}{
   \begin{list}{$\bullet$}
    { \setlength{\itemsep}{0pt}
      \setlength{\parsep}{2pt}
      \setlength{\topsep}{0pt}
      \setlength{\partopsep}{0pt}
      \leftmargin=25pt
\rightmargin=0pt
\labelsep=5pt
\labelwidth=10pt
\itemindent=0pt
\listparindent=0pt
\itemsep=\parsep
    }
}
\newcommand{\squishend}{\end{list}}

% use extensively to toggle between paper and TR
\newcommand{\eat}[1]{}
% \newcommand{\papertext}[1]{{\leavevmode\color{blue}{#1}}}
% \newcommand{\techreport}[1]{{\leavevmode\color{red}{#1}}}
\newcommand{\papertext}[1]{#1}
\newcommand{\techreport}[1]{}
\newcommand{\boldpara}[1]{\textbf{\paragraph{#1}}}
% de-facto paragraph format
\newcommand{\stitle}[1]{\par\noindent\textbf{#1}}
\newcommand{\tvcg}[1]{{\leavevmode\color{blue}{#1}}}
\newcommand{\cut}[1]{{\leavevmode\color{lightgray}{#1}}}
\newcommand{\ccut}[1]{} %confirmed cut
\def\plainauthor{Doris Jung-Lin Lee, Aditya Parameswaran}
\def\emptyauthor{} 
\def\plainkeywords{Data visualization, exploratory data analysis, visual querying.}
\def\plaingeneralterms{Documentation, Standardization}

\newcommand{\zv}{\textit{zenvisage}\xspace}
\newcommand{\sbd}{\textsc{Storyboard}\xspace}
\newcommand{\agp}[1]{\textcolor{green}{Aditya: #1}}
\newcommand{\dor}[1]{\textcolor{blue}{Doris: #1}} 

\newcommand\notes[1]{\textcolor{red}{#1}}

% To make various LaTeX processors do the right thing with page size.
\def\pprw{8.5in}
\def\pprh{11in}
\special{papersize=\pprw,\pprh}
\setlength{\paperwidth}{\pprw}
\setlength{\paperheight}{\pprh}
\setlength{\pdfpagewidth}{\pprw}
\setlength{\pdfpageheight}{\pprh}
%%%%%%%%%%%%%%%%%%%%%%%%%%%%%%%%%%%%%%%%%%%%

\begin{document}
\title{Towards a Holistic Workflow for Visual Data exploration}
\author{\plainauthor\\
\{jlee782,adityagp\}@illinois.edu\\
University of Illinois, Urbana-Champaign}
\maketitle
\begin{abstract}
Visualizations is one of the most effective and common data ---widespread use in --- a variety of domains. 

The challenges includes ------.  We describe the exemplary systems -- drawing from our --- both --- ongoing research --future directions and opportunties in this space.

% This paper surveys the emerging field of formally reasoning
% about and optimizing open-ended crowdsourcing, a popular and crucially important, but severely
% understudied class of crowdsourcing—the next frontier in crowdsourced data management. The underlying
% challenges include distilling the right answer when none of the workers agree with each other,
% teasing apart the various perspectives adopted by workers when answering tasks, and effectively selecting
% between the many open-ended operators appropriate for a problem. We describe the approaches that
% we’ve found to be effective for open-ended crowdsourcing, drawing from our experiences in this space.
\end{abstract}

% Eventually, we need to compile all of these into one single file  during submission time. 
%!TEX root = main.tex
\section{Introduction}% 
\par With the ever-increasing complexity and size of datasets, there is a growing demand for information visualization tools that help analysts make sense of large amounts of data. Visual data exploration is one of the most commonly used technique that powerfully couples human insights with statistical and data management solutions to help analysts rapidly generate hypothesis, discover interesting trends and patterns, and identify clusters and anomalies. In this paper, we present challenges in the various stages in the cycle of visual analysis, and discuss ongoing and future directions of research that addresses these issues.
\par Our paper focuses on ongoing research in interactive systems that guides users through the acts of visual data exploration as shown in Figure~\ref{fig:cycle}, including searching for a precise visualizations, querying with a vague intent, and recommending visualizations that facilitates better data understanding and awareness. Our presentation of the spiral in Figure~\ref{fig:cycle} does not imply a monotonic sequences of actions, nor are the acts of visual data exploration mutually exclusive from one another, instead the diagram should read as a series of workflow that interleaves and forms an iterative cycle that reinforces one another. For example, as we will discuss in Section \ref{sec:hypothesis}, better data understanding can lead to better precise visual querying behaviors.
\begin{figure}[h!]
\label{fig:table}
\centering
\includegraphics[width=0.7\linewidth]{figures/table.png}
\caption{Overview of the systems described in this paper. Columns are organized into discovery goals and rows are ordered by decreasing levels of query specificity and correspondingly increasing levels of autonomous assistance.}
\end{figure}
\par On the horizontal axis of Figure \ref{fig:table}, we list five different common discovery goals in visual data exploration characterized by functionalities in existing systems and related works on visualization taxonomy\cite{Heer2012,Amar2005}. The table omits the description of low-level task (such as filtering, sorting), since these functionalities are seen as commonplace in popular data analytics tools such as Excel, Tableau. On the vertical axis, we describe systems with varying degree of query specificity. All discovery goals except for summarization and explaination could be performed through exact and complete specification to query languages and programmatic APIs to statistical tools. Since these systems are often not easily accesible by non-experts, in Section \ref{sec:precise}, we introduce interactive systems that allows users to perform visual search of desired patterns, which in turn facillitates other highlighted discovery goals. Section~\ref{sec:vague} describes intelligent systems that infers users intent from complex and imprecise queries. The bottom-most row of the figure refers to recommendation systems that guide users towards better data understanding with minimial required inputs, as described in Section~\ref{sec:understanding}.
\dor{INCOMPLETE: Need to substitute the old story of cycle of exploration below to the new 2D table story above}
% \begin{figure}[h!]
% \label{fig:cycle}
% \centering
% \includegraphics[width=0.4\linewidth]{figures/cycle.pdf}
% \caption{Cycle of visual data exploration.}
% \end{figure}
% “What is the problem”, “Why is this important”, “Why do previous approaches fail”, “Why is this hard”, “What is our approach” paragraphs into the intro
\par Starting from the innermost spiral, we discuss how precise visual query systems help accelerate the process of finding desired visualizations. Precise visual query system addresses the common problem of having to manually examine large numbers of visualizations in search of a desired pattern, which can be error-prone and inefficient. We discuss our work on \zv which allows analysts to specify their desired visualization through front-end interactions and automatically returns a ranked list of visualizations that closely matches with the input query.
\par In Section~\ref{sec:precise}, we highlight examples from our \zv design study where precise querying alone is insufficient for addressing all the visual querying demands required in real-world use cases. In addition, we find that users often do not have a good idea of what they want to query for without looking at example visualizations or summaries of the data. To improve the flow of visual data exploration, we advocate to make visual query system more expressive by supporting a wider class of vague queries (Section~\ref{sec:vague}) and make it easier to know what to query by recommending visualizations that facilitate data awareness (Section~\ref{sec:understanding}).
\par Section~\ref{sec:vague} explains why the trade-off between expressiveness and usability in most interactive analytics system is an inherent challenge in making visual query systems more expressive. We discuss a growing class of \textit{intelligent visual querying system} (IVQS) that tries to interpret the `vagueness' of queries and allow users to tweak or refine their queries through a feedback mechanism.
\par To address the problem of guiding users to portions of the data that they might be interested in querying, Section~\ref{sec:understanding} introduces systems that help users become more aware of their dataset and visualize where they are in their analysis workflow. The challenge in building these systems involves understanding what types of visualizations should be recommended to facilitate data awareness. As an example, we describe our work on \sbd, a system that provides data summaries and guides users through informative subsets of data. Finally, we discuss related works on how visualizing provenance and situational information can guide users towards more informative analysis actions.
\par Our paper mainly focusses on the search and recommendation of visualization for data exploration, but also touches on related work in generic data querying. Related works on recommendation and automatic selection of the visualization design and encoding~\cite{Wongsuphasawat2017,Mackinlay2007} as well as frameworks and grammar for interactive visualization (Vega, D3) is out of the scope of this paper.

% graphical presentation (ShowMe, Voyager, etc) as well as declarative specification of interactive visualization (Vega-lite, D3 papers, cite Heer and Shneiderman) is out of scope of this ---.  
% For a more detailed survey --- , see ----\cite{Heer2012,}. 
%, so that analysts can make more informed analysis decisions subsequently
% Visualization ---> insights. 
% \begin{enumerate}
% 	\item Visual analysis help discover insights, etc.
% 	\item supporting cycle of visual data exploration. 
% 	\item In this paper, we introduce various challenges in the ---- visual analysis, and discuss ---- solutions ---.
% 	\item Challenge \#1: Precise search
% 		\item finding the right vis and relevant data is hard, 
% 		\item  we discuss \zv as one use case and solution in this space (Section \ref{sec:precise})
% 	\item Challenge \#2: Need for in-the-loop support. how do people query visually? 
% 		\item During our PD + others, hypothesis + in the loop considerations is important. We discuss relevant work and our findings in Section \ref{sec:hypothesis}.
% 		\item Need to formulate complex expressive queries, language is good, but need interface (e..g visual primitives) to support this. (Section \ref{sec:vague})
% 		\item top-down, bottom-up --> point to need for bottom-up recommendations (Section \ref{sec:understanding})
% 	\item Challenge \#3: Vague and intelligent search 
% 		\item Users might not always have something they want to start with, supporting vague querying (Section~\ref{sec:vague})
% 	\item Challenge \#4: Cold-start recommendation for data understanding. One aspect of this is to gain understanding of dataset, (e.g. representative and outliers in \zv, bottom up exploration)we discuss \sbd as one example solution addressing this problem (Section \ref{sec:understanding}).
% \end{enumerate}{}
%, including searching for patterns, identifying clusters and anomalies, compare, summarize, and explain, ince we wanted to --- high-level goals ---discovery, 

%!TEX root = main.tex
\section{Precise Visual Querying\label{sec:precise}}
Visual analysis is valuable often reveal important -----. However, there are many -----. 

\subsection{Motivating Example}
Astronomers from the The Dark Energy Survey (DES)\cite{Drlica-Wagner2017} are interested in finding anomalous time series to discover astrophysical transients (objects whose brightness changes dramatically as a function of time), such as supernova explosions or quasars. When trying to find celestial objects corresponding to supernovae, which have a specific pattern of brightness over time, scientists need to individually inspect the visualizations of each object until they find ones that match the pattern. With more than 400 million objects in their catalog, each having their own set of time series brightness measurement, the process of manually exploring a large number of visualizations is not only error-prone, but also overwhelming for scientists who do not have extensive knowledge about their dataset.  

% Intention driven task-based querying (Precise search)
\subsection{Challenges}
\par The astronomy use case highlights a common challenge in exploratory data analysis (EDA). There is often a large space of possible visualizations that could be generated from a given dataset and manual search through this large collection is inefficient.
\par There has been many related work in this ------- varying different space of possible visualizations, including visual encodings------data facets, . We will focus on 


Data is agnostic to the user, intention ---, by building tools---, Section \ref{sec:precise} to \ref{sec:vague} have focussed on extracting what user want from data. bridging together what user want from data, what data has to offer, supporting interactive discourse between the two. 

There’s a large space of possibilities, manual search is tedious. Either using one-size-fits-all statistics, templates, heuristics as a solution or problem only applicable to a subset of analytic tasks\cite{Vartak2015,Vartak2017}. Propose VQS as a solution\cite{Lee2017}. 
\subsubsection{Usage Scenario}
\subsection{Effortless Data Exploration with \zv}
\begin{itemize}
	\item ZQL offers a way to iterate over collections of visualizations\cite{Wongsuphasawat2016}. Iterate over collections of visualizations \cite{Siddiqui}
	\item 
\end{itemize}

\subsection{Challenges Ahead}
The goal here is to help novice submit precise queries without SQL background, easy to use interface. Our study found that VQS does more than just this, but still not enough.
\begin{itemize}
	\item Precise Search Fail to understand intricacies of user need/intent, need more expressivity/flexibility for querying.
	\item  No perfect training workload, real-world data + task is noisy and complex. 
	\item towards more holistic model for insight discovery
\end{itemize}

%!TEX root = main.tex
\section{Hypothesis Formation\label{sec:hypothesis}}
\subsection{Supporting cycle of visual analysis}
\begin{itemize}
	\item Essential ingredient in facilitating intelligent vague querying and exploration.
	\item This is a human process (\cite{Heer2012,Pirolli})
	\item Iterative Hypothesis Exploration/Refinement : argue that the following properties is important to sustain this “cycle of visual analysis” 
\end{itemize}
\subsection{Visual Querying in the framework of Data Sensemaking}
\cite{Lee2017}
Our participatory design ---situating visual querying in the framework of ------. Our \zv work ------- Bottom up and top down querying in VQS facilitates rapid insight discovery. More importantly pointed towards a need for vague querying. Give some examples of vague querying.

\subsection{Towards 3‘I’s of rapid hypothesis generation support}
Given our observations from the participatory design study, we distill several desiderata for the next generation VQSs. 
Towards 3‘I’s Interactive, Iterative, Informative (Give examples from the ZV-TVCG paper)
\stitle{Interactive flow}:
 (how natural is it to move between analysis steps, facilitate fluid analysis and not get “stuck”) : interactivity, feedback (latter is quite unexplored), and recommendation, expressivity (how easy is it to express what to do via interactions) and diversity of actions that could be performed.
\stitle{Iterative}: query refinement, dialogue (not a one-shot query)
Joining the flow: Section \ref{sec:vague} focusses on the first two items .
\stitle{Informative}: not just task-based interestingness but more explanation-based (causality, introduce distribution awareness notion in viz-sum), focussed on data understanding, which we will discuss in Section \ref{sec:understanding}
%!TEX root = main.tex
\section{Vague Intelligent Search\label{sec:vague}}
Accounting for user interaction, mental models. More global objective taking into account user with the goal of dataset understanding rather than task completion.
\subsection{Challenges}
\begin{itemize}
\item Inferring user intent in querying and context is important (both in terms of user input and what is recommended)
\item tools can not assume user has querying intention. exploration without intention, user don’t know what they are searching for --> Recommendation.
\item The important thing here is identifying what should be done by the system v.s. requested from user. Inappropriate choice of these will result in lack of expressibility and user feeling lack of control of analysis, limiting exploration.
\item Need for a unified framework of inference to take all of these into account (e.g. natural language, etc)
\end{itemize}
\subsection{\sbd: Navigating Through Data Slices with Hierarchical Summary of Visualizations}
%!TEX root = main.tex
\section{Towards Dataset Understanding\label{sec:understanding}}
\subsection{Challenges}
\begin{itemize}
\item Problem of cold-start recommendation (as discussed earlier use may not always know what to query for)
\item related works have focussed on making specification easier , but not really trying to understnad user intent or what might the user want to see.
\item Within a dataset, structure and provenance is essential to help users navigate and provide users with sense of coverage and completion. This is an important but underexplored area. (viz-sum, Sarvghad et al 2017)
\item provenance of schema and attribute understanding (coverage, etc) 
\end{itemize}

\subsection{Examples}
\begin{itemize}
	\item understanding distributions (distribution awareness)
	\item providing overview recommendations (representative trends and outliers)
\end{itemize}

\subsection{\sbd: Navigating Through Data Slices with Hierarchical Summary of Visualizations}

\begin{figure}[h!]
\label{fig:modalities}
\centering
\includegraphics[width=0.7\linewidth]{figures/storyboard.pdf}
\caption{Example dashboard with generated from the Police Stop Dataset \cite{police}, which contains records of police stops that resulted in a warning, ticket, or an arrest.The attributes in the dataset include driver gender, age, race, and the stop time of day, whether a search was conducted, and whether contraband was found. We generate a dashboard of visualizations with bar charts with x-axis as the stop outcome (whether the police stop resulted in a ticket, warning, or arrest/summons) and y-axis as the percentage of police stops that led to this outcome.}
\end{figure}

Data is agnostic to the user, intention ---, by building tools---, Section \ref{sec:precise} to \ref{sec:vague} have focussed on extracting what user want from data. bridging together what user want from data, what data has to offer, supporting interactive discourse between the two. 
 Either using one-size-fits-all statistics, templates, heuristics as a solution or problem only applicable to a subset of analytic tasks\cite{Vartak2015,Vartak2017}. 

%!TEX root = main.tex
\section{Concluding Remarks\label{sec:conclusion}}

% Data is agnostic to the user, intention ---, by building tools---, Section \ref{sec:precise} to \ref{sec:vague} have focussed on extracting what user want from data. bridging together what user want from data, what data has to offer, supporting interactive discourse between the two. 
%  Either using one-size-fits-all statistics, templates, heuristics as a solution or problem only applicable to a subset of analytic tasks\cite{Vartak2015,Vartak2017}. 

In this paper, we advocate supporting a desiderata consisting of  3`I's in the cycle of visual data exploration:
\begin{itemize}
	\item \textbf{Informative}: Section~\ref{sec:precise} discusses how precise visual query systems provide informative visualizations to accelerate the process of data discovery. 
	\item \textbf{Interactive and iterative}: Section~\ref{sec:vague}
	Joining the flow, query refinement, dialogue (not a one-shot query), feedback and recommendation, expressivity (how easy is it to express what to do via interactions) and diversity of actions that could be performed.
	\item \textbf{Integrated}: Section \ref{sec:understanding} discusses the challenges and opportunities in moving from intention-driven querying to facilitating more integrated data understanding and awareness during the analysis workflow.
\end{itemize}

{\footnotesize \bibliographystyle{named}
\bibliography{reference}}
\end{document}
