%!TEX root = main.tex
\section{Introduction}
% Visualization ---> insights. 
% \begin{enumerate}
% 	\item Visual analysis help discover insights, etc.
% 	\item supporting cycle of visual data exploration. 
% 	\item In this paper, we introduce various challenges in the ---- visual analysis, and discuss ---- solutions ---.
% 	\item Challenge \#1: Precise search
% 		\item finding the right vis and relevant data is hard, 
% 		\item  we discuss \zv as one use case and solution in this space (Section \ref{sec:precise})
% 	\item Challenge \#2: Need for in-the-loop support. how do people query visually? 
% 		\item During our PD + others, hypothesis + in the loop considerations is important. We discuss relevant work and our findings in Section \ref{sec:hypothesis}.
% 		\item Need to formulate complex expressive queries, language is good, but need interface (e..g visual primitives) to support this. (Section \ref{sec:vague})
% 		\item top-down, bottom-up --> point to need for bottom-up recommendations (Section \ref{sec:understanding})
% 	\item Challenge \#3: Vague and intelligent search 
% 		\item Users might not always have something they want to start with, supporting vague querying (Section~\ref{sec:vague})
% 	\item Challenge \#4: Cold-start recommendation for data understanding. One aspect of this is to gain understanding of dataset, (e.g. representative and outliers in \zv, bottom up exploration)we discuss \sbd as one example solution addressing this problem (Section \ref{sec:understanding}).
% \end{enumerate}{}
In this paper, we adovocate supporting a desiderata consisting of  3`I's in the cycle of visual data exploration:
\begin{itemize}
	\item \textbf{Informative}: Section~\ref{sec:precise} discusses how precise visual query systems provide informative visualizations to accelerate the process of data discovery. 
	\item \textbf{Interactive and iterative}: Section~\ref{sec:vague}
	Joining the flow, query refinement, dialogue (not a one-shot query), feedback (latter is quite unexplored), and recommendation, expressivity (how easy is it to express what to do via interactions) and diversity of actions that could be performed.

	\item \textbf{Integrated}: Section \ref{sec:understanding} discusses the challenges and opportunities in moving from intention-driven querying to facillitating more integrated data understanding and awareness during the analysis workflow.
\end{itemize}