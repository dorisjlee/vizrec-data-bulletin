%!TEX root = main.tex
\section{Introduction}% 

\par 
With the ever-increasing complexity 
and size of datasets,
there is a growing demand for 
information visualization tools
that can help data scientists make sense of large
volumes of data.
Visualizations help discover 
trends and patterns, 
spot outliers and anomalies, 
and generate or verify hypotheses.
Moreover, 
visualizations are visceral and intuitive: 
they tell us stories about our data; 
they educate, delight, inform, 
enthrall, amaze, and clarify.
This has led to the overwhelming popularity
of point-and-click visualization tools like Tableau~\cite{Stolte2002},
as well as programmatic toolkits like ggplot, D3, Vega, and matplotlib. 
We term these tools as {\em visualization-at-a-time} approaches, since
data scientists need to individually 
generate each visualization (via code or interactions),
and examine them, 
one at a time.


\par
As datasets grow in size and complexity, 
these visualization-at-a-time approaches start to break down,
due to the limited time availability on the 
part of the data scientists---there 
are often too many visualizations to examine for a given 
task, such as identifying outliers, or inferring patterns. 
Even on a single table, 
visualizations can be generated
by varying the subsets of data operated on, 
or the attributes (or combinations
thereof) that can be visualized. 
If we add in various visualization modalities, encodings,
aesthetics, binning methods, and transformations,
this space becomes even larger.


\par
Thus, there is a pressing need for an 
intelligent,
interactive, understandable, usable, and
enjoyable tool that can help 
data scientists navigate
collections of visualizations.
We term our hypothesized tool \vida,
short for {\em VIsual Discovery Assistant}.
Data scientists specify their discovery
goal at a high level,
with \vida 
automatically 
traversing visualizations to provide
solutions or partial solutions for the
specified discovery goal, thereby
eliminating the tedium and wasted
labor of comparable visualization-at-a-time 
approaches.

\par
\stitle{\vida Dimensions.} In order to be a holistic solution for 
navigating collections of visualizations,
\vida must be able to support various discovery
settings. 
We organize these settings along two
dimensions, displayed along the columns and rows (respectively) of Table~\ref{fig:table}---first, the overall discovery goal,
and second, the degree of specificity of the goal.
We also cite references (described later on)
for systems that partially provide the necessary
functionality for the given setting.

\par
 We identify five 
common discovery goals in visual data exploration, organized
along the columns:
{\em finding patterns}, {\em identifying anomalies/clusters}, {\em summarizing}, 
{\em performing comparisons}, {\em providing explanations}.
These five goals borrow borrow from functionalities in existing
systems, as well as related visualization task taxonomies~\cite{Heer2012,Amar2005}.
We omit low-level goals such as filtering or sorting, since
these functionalities are common in 
visualization-at-a-time tools and toolkits.
We also omit \agp{more here}

\par 
 We identify three degrees of specificity
for the discovery goal, organized along the rows:
{\em precise}, {\em fuzzy}, {\em underspecified}.
The degree of specificity characterizes the division
of work between how much user has to specify
versus how much the system has to automatically
infer and aid in accomplishing the discovery goal. 
At the topmost row, the onus is placed on the user
to provide an exact and complete specification of 
what the solution to their discovery
goal must look like;
at the middle row, the user can provide
a vague specification of what the solution must look like;
and finally, at the bottom row,
the user provides a minimal specification, or
leaves the characteristics of the solution underspecified,
leaving it up to the system to ``fill in'' the rest.
Naturally, as we proceed down the rows,
it gets harder for the system to automatically
interpret what the user might have in mind as a solution
to their discovery goal.

\agp{Describe what precise, fuzzy, and underspecified are.}
\agp{got until here.}

\newpage





 At the topmost row, we find that while most discovery goals can be accomplished through exact and complete specification to query languages and programmatic APIs to statistical tools, these systems are often not easily accesible by non-experts.
% On the vertical axis, we describe systems with varying degree of query specificity. All discovery goals except for summarization and explaination could be performed through exact and complete specification to query languages and programmatic APIs to statistical tools. 
\par In Section \ref{sec:precise}, we discuss how precise visual query systems help accelerate the process of finding desired visualizations, which in turn facillitates other highlighted discovery goals. Precise visual query system addresses the common problem of having to manually examine large numbers of visualizations in search of a desired pattern, which can be error-prone and inefficient. We discuss our work on \zv which allows analysts to specify their desired visualization through front-end interactions and automatically returns a ranked list of visualizations that closely matches with the input query.
\par Examples from our \zv design study demonstrates that precise querying alone is insufficient for addressing all the visual querying demands required in real-world use cases. In addition, we find that users often do not have a good idea of what they want to query for without looking at example visualizations or summaries of the data. To bridge the gap between user's high-level intent and what the system operates as inputs, we advocate that future research needs to look beyond simple precise visual querying by : 1) making visual query system more expressive by supporting a wider class of vague queries (Section~\ref{sec:vague}) and 2) making it easier to know what to query by recommending visualizations that facilitate data awareness (Section~\ref{sec:understanding}).
\par Accordingly, the next row in the table highlights a growing class of \textit{intelligent visual querying system} (IVQS) that interprets the `vagueness' of queries and allow users to tweak or refine their queries through a feedback mechanism. \dor{I think we need to expand this definition depending on the new content that we will be adding, ShapeSearch, SeeDB, Scorpion?} 
\par To address the problem of guiding users to portions of the data that they might be interested in querying, Section~\ref{sec:understanding} introduces systems that help users become more aware of their dataset and visualize where they are in their analysis workflow. The challenge in building these systems involves understanding what types of visualizations should be recommended to facilitate data awareness. As an example, we describe our work on \sbd, a system that provides data summaries and guides users through informative subsets of data. Finally, we discuss related works on how visualizing provenance and situational information can guide users towards more informative analysis actions.
%The bottom-most row of the figure refers to recommendation systems that guide users towards better data understanding with minimial required inputs, as described in Section~\ref{sec:understanding}.
% \begin{figure}[h!]
% \label{fig:cycle}
% \centering
% \includegraphics[width=0.4\linewidth]{figures/cycle.pdf}
% \caption{Cycle of visual data exploration.}
% \end{figure}
% “What is the problem”, “Why is this important”, “Why do previous approaches fail”, “Why is this hard”, “What is our approach” paragraphs into the intro
% \par Starting from the innermost spiral, 
% \par In Section~\ref{sec:precise}, we highlight examples from our \zv design study where precise querying alone is insufficient for addressing all the visual querying demands required in real-world use cases. In addition, we find that users often do not have a good idea of what they want to query for without looking at example visualizations or summaries of the data. To improve the flow of visual data exploration, we advocate to make visual query system more expressive by supporting a wider class of vague queries (Section~\ref{sec:vague}) and make it easier to know what to query by recommending visualizations that facilitate data awareness (Section~\ref{sec:understanding}).
% \par Section~\ref{sec:vague} explains why the trade-off between expressiveness and usability in most interactive analytics system is an inherent challenge in making visual query systems more expressive. We discuss a growing class of \textit{intelligent visual querying system} (IVQS) that tries to interpret the `vagueness' of queries and allow users to tweak or refine their queries through a feedback mechanism.
% \par To address the problem of guiding users to portions of the data that they might be interested in querying, Section~\ref{sec:understanding} introduces systems that help users become more aware of their dataset and visualize where they are in their analysis workflow. The challenge in building these systems involves understanding what types of visualizations should be recommended to facilitate data awareness. As an example, we describe our work on \sbd, a system that provides data summaries and guides users through informative subsets of data. Finally, we discuss related works on how visualizing provenance and situational information can guide users towards more informative analysis actions.
\par Our paper mainly focusses on the search and recommendation of visualization for data exploration, but also touches on related work in generic data querying. Related works on recommendation and automatic selection of the visualization design and encoding~\cite{Wongsuphasawat2017,Mackinlay2007} as well as frameworks and grammar for interactive visualization (Vega, D3) is out of the scope of this paper.

% graphical presentation (ShowMe, Voyager, etc) as well as declarative specification of interactive visualization (Vega-lite, D3 papers, cite Heer and Shneiderman) is out of scope of this ---.  
% For a more detailed survey --- , see ----\cite{Heer2012,}. 
%, so that analysts can make more informed analysis decisions subsequently
% Visualization ---> insights. 
% \begin{enumerate}
% 	\item Visual analysis help discover insights, etc.
% 	\item supporting cycle of visual data exploration. 
% 	\item In this paper, we introduce various challenges in the ---- visual analysis, and discuss ---- solutions ---.
% 	\item Challenge \#1: Precise search
% 		\item finding the right vis and relevant data is hard, 
% 		\item  we discuss \zv as one use case and solution in this space (Section \ref{sec:precise})
% 	\item Challenge \#2: Need for in-the-loop support. how do people query visually? 
% 		\item During our PD + others, hypothesis + in the loop considerations is important. We discuss relevant work and our findings in Section \ref{sec:hypothesis}.
% 		\item Need to formulate complex expressive queries, language is good, but need interface (e..g visual primitives) to support this. (Section \ref{sec:vague})
% 		\item top-down, bottom-up --> point to need for bottom-up recommendations (Section \ref{sec:understanding})
% 	\item Challenge \#3: Vague and intelligent search 
% 		\item Users might not always have something they want to start with, supporting vague querying (Section~\ref{sec:vague})
% 	\item Challenge \#4: Cold-start recommendation for data understanding. One aspect of this is to gain understanding of dataset, (e.g. representative and outliers in \zv, bottom up exploration)we discuss \sbd as one example solution addressing this problem (Section \ref{sec:understanding}).
% \end{enumerate}{}
%, including searching for patterns, identifying clusters and anomalies, compare, summarize, and explain, ince we wanted to --- high-level goals ---discovery, 



\begin{table}[!b]
\scriptsize
\begin{tabular}{l|lllll}
                                & \multicolumn{5}{c}{Discovery Goals}                                                                                                                                                              \\ \hline
                                & Find Patterns                       & Identify Anomalies/Clusters                    & Compare                                         & Summarize & Explain                                  \\ \hline
                                & \multicolumn{3}{c|}{\cellcolor[HTML]{CBCEFB}Query Languages + Statistical APIs}                                                            &           &                                          \\ \hline
Precise Visual Query System     & \multicolumn{3}{c|}{\cellcolor[HTML]{FFCE93}Zenvisage}                                                                                     &           &                                          \\ \hline
Intelligent Visual Query System & \cellcolor[HTML]{9AFF99}ShapeSearch & \cellcolor[HTML]{FFFC9E}Scorpion, Natural Language & \cellcolor[HTML]{FFFC9E}SeeDB, Natural Language &           & \cellcolor[HTML]{FFFC9E}Natural Language \\ \hline
Recommendation Guidance System  &                                     & \multicolumn{4}{c}{\cellcolor[HTML]{96FFFB}Storyboard}                                                                                                     
\end{tabular}
\caption{Overview of the systems described in this paper. Columns are organized into discovery goals and rows are ordered by decreasing levels of query specificity and correspondingly increasing levels of autonomous assistance.}\label{fig:table}
\end{table}
