%!TEX root = main.tex
\section{Concluding Remarks\label{sec:conclusion}}
\par Data is inherently agnostic to the diverse information needs that any particular user may have. Visual data exploration systems help bridge the gap between what users want to get from the data through querying and what insights the data has to offer through recommendations. To facilitate a more productive collaboration, in this paper, we outline our vision for \vida motivated by related works from precise visual querying of informative visualizations to accelerate the process of data discovery, to interpreting vague and complex query through query refinement feedback, to recommendations that promote better distributional and contextual awareness for users. We hope that the agenda sketched out in this paper sheds light to the many more exciting research questions and opportunities to come in this nascent growing field.

% and encourage database researchers to ----- tackle in this space.
% This paper discusses how to better support these and how ----. Section~\ref{sec:precise} discusses how precise visual query systems provide informative visualizations to accelerate the process of data discovery. Section~\ref{sec:vague} discusses mixed-intiative systems that accounts for vague and complex query and allow users to provide feedback to refine their queries. Section \ref{sec:understanding} discusses the challenges and opportunities in moving from intention-driven querying to facilitating more integrated data understanding and awareness during the analysis workflow.

% Data is agnostic to the user, intention ---, by building tools---, Section \ref{sec:precise} to \ref{sec:vague} have focussed on extracting what user want from data. bridging together what user want from data, what data has to offer, supporting interactive discourse between the two. 

% \par In this paper, we advocate supporting a desiderata consisting of  3`I's in the cycle of visual data exploration:
% \begin{itemize}
% 	\item \textbf{Informative}: Section~\ref{sec:precise} discusses how precise visual query systems provide informative visualizations to accelerate the process of data discovery. 
% 	\item \textbf{Iterative}: Section~\ref{sec:vague} discusses mixed-intiative systems that accounts for vague and complex query and allow users to provide feedback to refine their 
% 	Joining the flow, query refinement, dialogue (not a one-shot query), feedback and recommendation, expressivity (how easy is it to express what to do via interactions) and diversity of actions that could be performed.
% 	\item \textbf{Integrated}: Section \ref{sec:understanding} discusses the challenges and opportunities in moving from intention-driven querying to facilitating more integrated data understanding and awareness during the analysis workflow.
% \end{itemize}

%  Either using one-size-fits-all statistics, templates, heuristics as a solution or problem only applicable to a subset of analytic tasks\cite{Vartak2015,Vartak2017}. 
