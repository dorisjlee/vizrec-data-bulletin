%!TEX root = main.tex

\section{The Precise Setting: Existing Visual Query Systems}\label{sec:precise}
While visual data exploration often reveals 
important anomalies or trends 
in the data~\cite{Heer2012,Morton2014}, 
it is challenging to 
use visualization-at-a-time systems to 
repeatedly choose the right subsets of 
data and attributes to visualize
in order to identify desired insights.
We first motivate the precise setting through a use-case from astronomy.

\subsection{Motivating Example: Discovering Patterns in Astronomical Data}
Astronomers from the Dark Energy Survey (DES) 
are interested in finding 
anomalous time series 
in order to discover 
transients, 
i.e., objects whose brightness 
changes dramatically as a function of time, 
such as supernova explosions or quasars~\cite{Drlica-Wagner2017}. 
When trying to find celestial objects 
corresponding to supernovae, 
with a specific pattern of brightness over time, 
astronomers individually inspect the corresponding 
visualizations until 
they find ones that match the pattern. 
With more than 400 million objects in their catalog, 
each having their own set of time series brightness measurements, 
manually exploring so many 
visualizations is not only error-prone, 
but also overwhelming.
Many astronomers instead rely on guess-work 
or prior evidence to explore the visualizations,
rather than directly ``searching'' for the patterns
that they care about. 
While most astronomers do use 
programming tools, such as Python and R,
it is cumbersome to express each new information need
via carefully hand-optimized code. 
The astronomy use case highlights a 
common challenge in visual data exploration,
where there are often a large number of visualizations
for each discovery goal,
and manual exploration is impossible.
There is no systematic way to create, compare, filter,
and operate on large collections of visualizations.

\subsection{Ongoing Work and Research Challenges within the Precise Setting}

\stitle{Discovering Patterns and Anomalies/Clusters, and Performing Comparisons with \zv.}
\zv is an example of a VQS that operates on 
large collections of visualizations.
\zv is built on top of a visual query language
called ZQL, which operates on collections of visualizations, and returns
collections of visualizations,
using primitives such as visualization composition,
sorting, and filtering~\cite{Siddiqui2016}. 
In addition, via user-defined functions (also provided as built-ins),
ZQL can compute the distance for a visualization from another---and this
distance can be used as a building block for more complex
operations, such as the first three discovery goals listed
in Table~\ref{fig:table}, i.e., finding a visualization
matching a pattern, detecting an outlier or cluster centers,
or comparing visualizations with each other.
Thus, ZQL operates at a level higher than
languages for specifying visual encodings of
individual visualizations~\cite{Stolte2002,Wilkinson2005}.
ZQL can be used to construct a rich variety of queries,
including the following on a real-estate dataset:
\squishlist
	\item {\em Find Patterns.} Find visualizations of cities whose housing price over time is similar to Manhattan. 
	\item {\em Identify Anomalies.} Find visualizations of cities with both an increasing housing price trend and an increasing foreclosure rate trend.
	\item {\em Perform Comparisons.} Find visualizations for which New York and Champaign differ the most.
	\item {\em Find Patterns, followed by Clusters.} For cities that are similar
	to New York on sales price trends, find typical trends for foreclosure rates.
\squishend

\par While ZQL is a useful starting point, writing ZQL
queries can be daunting for users who are not comfortable
with programming.
Therefore, we extracted a typical workflow
of visual querying for finding patterns, identifying
outliers and clusters, and making comparisons, and made it expressible
via simple interactions. As shown in Figure~\ref{fig:modalities}, the user can input their search pattern via (a) inputting a set of points,
(b) a sketch, (c) dragging-and-dropping an existing visualization,
or (d) inputting an equation, 
and specify a space of visualizations over which to search.
The system returns a ranked list of matching visualizations, shown in the ``Results'' pane below the canvas in (b), for the corresponding ZQL query.
The system also provides typical trends 
(called ``Representative Patterns'', highlighted under (c)) and outliers
for additional context. 
The user can also switch to a tabular interface wherein
they can input ZQL queries.  
Users can also make comparisons between different 
subsets of the data by examining visualizations 
of dynamically created classes in \zv~\cite{Lee2017}.
Thus, this workflow captures one variation of the three
discovery goals---finding patterns, identifying outliers/clusters,
and making comparisons. 
\begin{figure}[!t]
\centering
\vspace{-10pt}
\includegraphics[width=0.8\textwidth,frame]{figures/modalities.pdf}
\caption{\zv offers a variety of querying modalities, including: a) uploading a sample pattern from an external dataset as a query, b) sketching a query pattern, c) dragging-and-dropping an existing pattern from the dataset, and d) inputting an equation as a query, from \cite{Lee2017}.}
\label{fig:modalities}
\vspace{-10pt}
\end{figure}


\stitle{Other Related Precise Visual Querying Systems.}
There has been a range of work primarily focused on locating visualizations
matching certain patterns, including the influential work
on TimeSearcher~\cite{hochheiser2004dynamic}, where the query is
specified as a box constraint, and 
Query-by-Sketch~\cite{wattenberg2001sketching}, where the query is 
sketched as a pattern. 
This was applied to search queries within Google Correlate~\cite{mohebbi2011google}. 
Recent work has also extended this to querying using regular
expressions~\cite{Zgraggen2015}, while
others have analyzed the semantics
of sketches~\cite{correll2016semantics,Mannino2018}.

\stitle{Limitations within the Precise Setting.}
The most important challenge within the precise setting
is the following: {\em How do we expose the entire capabilities of
 \vidaql via intuitive interactions and discoverable
modules?} So far, most tools that we listed above
support a restricted set of interactions,
primarily focused on finding patterns via sketches or box constraints. 
\zv extends and builds on this 
by supporting outlier and representative trend discovery
from the front-end interface. 
Unfortunately, performing comparisons, deriving explanations, or summaries,
is hard to do interactively. 
In fact, deriving explanations or summaries
is hard to do even with the full power of ZQL, since
it does not support the synthesis of new information
that goes beyond simple processing of collections of visualizations. 
It remains to be seen if and how ZQL can be extended to meet these capabilities.
Moreover, even for sketching, it would be interesting to see if
we can support analogous pattern search mechanisms
for other visualization types, such
as scatterplots
or heatmaps.


\subsection{Limitations with the Precise Setting}
\par Even if we could address 
the issues within the precise setting
in the previous section,
there are some fundamental limitations
with the precise setting itself 
that need to be addressed within a full-fledged \vida.
We discuss these limitations
in the context of a participatory 
design process for the development 
of \zv, in collaboration
with scientists from astronomy, genetics, 
and material science~\cite{Lee2017}. 


\stitle{The Problem of Interpreting Ambiguous, High-level Queries.}
When users interact with \zv, 
they often translate their ambiguous, 
high-level questions 
into an plan that consists of multiple interactions 
to incrementally address their desired query. 
The expressiveness of such a system 
comes from the multiplicative effect 
of stringing together combinations of 
interaction sequences into a customized workflow. 
Designing features that diversify potential 
variations expands the space of possible workflows 
that could be constructed during the analysis. 
However, even with many supported interactions, 
there were still vague and complex queries 
that could not be decomposed 
into a multi-step interaction workflow. 
For example, \zv was unable to support 
high-level queries that involved the use of vague 
descriptors for matching to specific data 
characteristics, such as finding patterns that 
are `flat and without noise', or `exhibits irregularities'. 
These scenarios showcase examples of lexical ambiguity, 
where the system can not map the vague term `irregular' or `noisy' 
into the appropriate series of analysis steps 
required to find these patterns. 
In Section~\ref{sec:vague}, we survey the challenges 
in supporting vague and complex queries 
and point to some ongoing research.

\stitle{The Problem of Not Knowing What to Query.}
Another key finding is that 
users often do not start their analysis 
with a specific pattern in mind or a specific comparison to perform. 
For example, within \zv, many users 
first made use of the recommended 
representative trends and outlier visualizations 
as contextual information 
to better understand their data, 
or to query based on these recommended visualizations.
Thus, the recommended visualizations
are often used to form queries in a bottom-up
manner, rather than users coming up with a query
in a top-down, prescriptive manner.
Thus, there is a need for visualization recommendations~\cite{Vartak2017}
that can help users jump-start their exploration.
Recommendations help facilitate a smoother flow
of analysis by closing the gap between the two modalities
of querying and exploration, 
reminiscent of the browsing and searching 
behaviors on the Web~\cite{Olston2003}, thus 
ensuring that user is never stuck or out of 
ideas at any point during the analysis. 
Typically, visualization recommendations help 
accelerate the process of discovering 
interesting aspects of the data by broadening exploration. 
In Section~\ref{sec:minimal}, we argue 
that recommendations should not just focus on 
increasing breadth-wise exploration
so that other views of the data are discoverable,
but they should also contribute
towards helping users become more aware of the 
distributions and trends in the data,
as well as the broader context of their analysis.

% \agp{what does data discoverability mean?}\dor{I was more referring to how existing systems focusses on increasing breadth-wise exploration so that other views of the data is more discoverable. For data understanding, it's not just about the more views shown the better (e.g. univariate summaries), but considers the cognitive effects of showing particular recommendations (i.e. our \sbd objective, deviation-based interestingness metric in \seedb)}, but also contribute towards helping users become more aware of the distributions in their data and the context of their analysis.
% % \subsection{Top-down and Bottom-up Querying Modalities}
% % \par Our study also revealed the challenges of coming up with a query in a prescriptive manner. 
% % Need a topic sentence/paragraph
% \par Pirolli and Card's notional model distinguishes between information processing tasks that are \textit{top-down} (from theory to data) and \textit{bottom-up} (from data to theory)~\cite{Pirolli}. In the context of visual querying, users employ top-down approaches by starting with a hypothesis on what patterns to look for and express it through sketching or inputting an equation (Figure~\ref{fig:modalities}b,d). On the other hand, bottom-up approaches originate from the data (or equivalently, the visualization). For example, the user may drag and drop a visualization of interest in the dataset as the input query or upload a visualization from an external dataset (Figure~\ref{fig:modalities}a,c). These two modalities of querying are reminiscent of the browsing and searching behaviors on the Web, which derives from the successful application of foraging theory to Web Search~\cite{Olston2003}.
% \par While the usage of each querying feature may vary from one participant to the next, our interactions with the scientists showed that \emph{bottom-up querying via drag-and-drop was more intuitive and more commonly used than top-down querying methods when the users have no desired patterns in mind}, which is commonly the case for exploratory data analysis. One of the main reason why participants did not find sketching useful was that they often do not start their analysis with a pattern in mind. Later, their intuition about what to query is derived from other visualizations that they see in the PVQS, in which case it made more sense to query using those visualizations as examples directly. Similarly, while functional fitting is a common operation in scientific data analysis, querying by equation is also unpopular, since it is challenging to formulate functional forms in an prescriptive, ad-hoc manner without seeing what the common patterns in the dataset are. 
%  %(e.g. after a filter is applied)  (e.g. find visualizations that are similar to the one in the largest representative clusters). 
% \par As evident from the representative and outliers visualization recommendations in \zv, recommendations facilitate a smoother flow of analysis by closing the loop between the two modalities of querying and exploration, thus ensuring that user is never stuck or out of ideas at any point during the analysis. Typically, visualization recommendation systems seeks to accelerate the process of discovering interesting aspects of the data by broadening exploration. In Section~\ref{sec:understanding}, we advocate recommendation systems should not only focus on data discoverability aspect of exploration, but also contribute towards helping users become more aware of the distributions in their data and the context of their analysis.


%While this could be expressed through user-defined functions in the underlying querying language ZQL, the learning curve and engineering cost is high. %Our previous example of queries that could not be expressed in the \zv interface showcases one example of  %\dor{Not sure if its worthwhile to include a couple sentences about Tarique's regex/NLP work here. e.g. the `up and then down' example}. 
%Participants in our \zv design study often created unexpected workflows that chained together analysis steps consisting of multiple interactions and controls. For example, geneticists in our study repeatedly explored representative trends to gain an overall sense of the typical profiles that exist in their dataset and queried mainly through drag-and-drop of these representative trends. Variations to their main workflow also include changing cluster sizes and display control settings to offer them different perspectives on the dataset. 