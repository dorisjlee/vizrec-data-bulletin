%!TEX root = main.tex
\section{The Fuzzy Setting: Towards Intelligent Visual Search}\label{sec:vague}
% \agp{We don't really say all that much about the 5 discovery goals
% in this section or the next. One option is to de-emphasize it throughout, or at least
% get rid of the table.}
% \dor{Is it sufficient highlight which discovery goals is covered by the tools that we discuss and how? It might also help to italicize and use a single consistent term for describing each discovery goal: \textit{Compare}, Identifying patterns as \textit{Diagonose}? , Cluster/Outliers as \textit{Overview}?, \textit{Summarize},\textit{Explain}}. 
\par As we argued in the previous section, 
there are ambiguous, high-level goals (e.g., 
``explain this bump in this visualization'', or ``find a product
for which the visualization of sales over time is bumpy'') that
cannot be captured within the precise setting. We will discuss this fuzzy setting in the following section. 


\subsection{The Challenge of Usability-Expressiveness Tradeoff}
\par The challenge for supporting ambiguous, 
high-level goals stems from the 
inevitable design trade-off between 
query expressiveness and interface 
usability in interactive data exploration 
systems~\cite{Jagadish2007,Morton2014}. 
This tradeoff is observed not only in 
visual data exploration systems, 
but also true for general ad-hoc data querying. 
While querying language such as SQL are highly expressive, 
formulating SQL queries that maps user's high-level intentions 
to specific query statements is challenging~\cite{Jagadish2007,Khoussainova2010}. 
As a result, query construction interfaces 
have been developed to address this issue by enabling 
direct manipulation of queries through 
graphical representations~\cite{Abouzied2012}, 
gestural interaction~\cite{Nandi2013}, and 
tabular inputs~\cite{Embley1989,Zloof1975}. 
For example, form-based query builders 
often consist of highly-usable interfaces 
that ask users for a specific set of information 
mapped onto a pre-defined query. 
However, form-based query builders 
are often based on query templates 
with limited expressiveness 
in their semantic and conceptual coverage, 
which makes it difficult for expert users 
to express complex queries. 
The extensibility of these systems 
also comes with high engineering costs, 
as well as potentially overwhelming users 
with too many potential options to chose from. 
Thus, there is a need for tools that enable users 
to formulate rich and complex queries, 
yet amenable to the fuzzy, imprecise query 
inputs that users might naturally formulate. 
%users mhighly usable even for novices.  
%Most systems design exhibits a trade-off between how expressive can the query be and how usable the interface is. 

% \begin{itemize}
% \item Inferring user intent in querying and context is important (both in terms of user input and what is recommended)
% \item tools can not assume user has querying intention. exploration without intention, user don’t know what they are searching for --> Recommendation.
% \item The important thing here is identifying what should be done by the system v.s. requested from user. Inappropriate choice of these will result in lack of expressibility and user feeling lack of control of analysis, limiting exploration.
% \item 
% \end{itemize}
%we can not assume a one-size-fit-all system that could fit the needs for users of different expertise levels and workloads. In this section, 

% \agp{One issue with the way this section is written is that the related work discussion is embedded within the research challenges. While this is fine, it deviates from the way the other sections are written. If there is a way we can make this uniform and consistent, that would be valuable.}
\subsection{Ongoing Work and Research Challenges within the Fuzzy Setting}
\par Given the tradeoff between 
expressiveness and usability, 
we discuss a growing class of VQSs that targets how to answer imprecise, fuzzy, and complex queries. In the fuzzy setting, the most important challenge is {\em how do we interpret fuzzy, complex queries,
and allow users to understand the results and refine
the analysis}. This boils down to several challenges 
that we will discuss in this section,
including:
{\em How can we develop better ways to
 resolve ambiguity by inferring 
 the information needs and 
 intent of users? 
 What is the appropriate level 
 of feedback and interactions for query refinement? 
 How can we develop interpretable 
 visual metaphors that explain how the query 
 was interpreted and why specific query results are returned?} 

We organize our discussion along the 
types of ambiguity that may arise  
in interpreting fuzzy, complex queries. 
Since systems 
targeting the fuzzy setting 
operate in the intermediate layer 
between users and \vidaql 
as shown in Figure~\ref{fig:vida_architecture}, 
we use the linguistic classification scheme 
to provide an analogy for where the ambiguous 
aspects of queries may arise, 
noting that the use of this analogy 
by no means limits our analysis to only natural language interfaces.


\stitle{Lexical Ambiguity}: 
Lexical ambiguity involves the use of 
vague descriptors in the input queries. 
Resolving these lexical ambiguities 
has been a subject of research in 
natural language interfaces for 
visualization specification\footnote{These are not VQSs, since
they do not operate on collections of visualizations,
rather, they expect one visualization to be the result for a query. Thus,
they act as a natural language interface for visualization-at-a-time systems.}, 
such as DataTone~\cite{Gao2015} and Eviza~\cite{Setlur2016}. 
These interfaces detect ambiguous quantifiers 
in the input query (e.g. ``Large earthquakes near California''), 
and then displays ambiguity widgets 
in the form of a widget to allow users 
to specify the definition of `large' 
in terms of magnitude and the number of miles 
radius for defining 
`near'\techreport{, as shown in Figure~\ref{fig:ambiguity}a}. 
These ambiguity widgets not only serve as a 
way to provide feedback to the system for 
lexically vague queries, 
but is also a way for explaining 
how the system has interpreted the input queries. 
It remains to be seen if similar ambiguity-resolution techniques
can be applied for exploration of collections of
visualizations as opposed to one visualization at a time.

In addition to ambiguity in the filter descriptions, 
as we have seen in the \zv study, 
there is often also ambiguity in the terminologies 
used for describing query patterns. 
\ssearch~\cite{Siddiqui2018} operates 
on top of an internal shape query algebra 
to flexibly match visualization segments, 
such as a trendline that ``first rises and then go down''. 
After the translation to the 
internal query representation, 
\ssearch then performs efficient 
perceptually-aware matching of visualizations. 
\ssearch is restricted to trendline pattern
search, without supporting the other four discovery goals.
Similarly, within \vida , 
we envision lexical ambiguity to be 
resolved by determining the 
appropriate \textit{parameters} 
to the internal \vidaql query for 
achieving the user's desired querying effects.

\stitle{Syntactic Ambiguity}: 
Syntactic ambiguity is related to the 
vagueness in specifying how the 
query should be structured or ordered. 
For example, DataPlay introduced 
the idea of syntax non-locality in SQL, 
in which switching from an existential 
(at least one) to a universal (for all) 
quantifier requires major structural changes 
to the underlying SQL query~\cite{Abouzied2012}. 
\techreport{As shown in Figure~\ref{fig:ambiguity}b, }
DataPlay consists of a visual interface 
that allowed users to directly manipulate 
the structure of the query tree into 
its desired specification. 
Within \vida, syntactic ambiguities resolution  h
involves mapping portions of the vague queries 
into to \textit{a series of multi-step workflows} 
to be executed in \vidaql and exposing feedback 
frameworks to allow users to directly tweak 
the internal query representation. 
% For example, ``show me profits over time
% of those products whose sales over time is noisy'',
% needs to be first executed as
%The query modification is done in a declarative manner in that the underlying mechanism in which the visualized workflow gets translated to the querying language is largely hidden from the end-user. 
\techreport{\agp{I don't understand this. Can you give me an example of where syntactic ambiguity may arise, e.g., in ZQL?} 
\dor{In ZQL, something along the line of searching for something increasing in one collection and decreasing in another, the ambiguity is in whether to loop through two collections simultaneously and sort by a common objective, or find the top k in one collection first, then lowest-k in another collection to find the desired visualization. Essentially syntactic ambiguity happens when there is ambiguity in how the sequence of operations should be carried out is vague or unspecified.}}

\stitle{Semantic Ambiguity}: Semantic ambiguity 
includes addressing `why' questions 
that are logically difficult to answer, 
such as `why do these datapoints look different from others?'. 
For example, Scorpion~\cite{Wu2013} visually traces the provenance 
of a set of input data tuples to find predicates 
that explain the outlier behavior;
it remains to be seen if this work, along with others
on causality, e.g., \cite{meliou2010causality,roy2014formal},
can be exposed via front-end interactions or natural language
queries. 
Semantic ambiguity can also arises 
when the user does not specify their 
intent completely or explicitly, 
which is often the case in early stages 
of visual data exploration. 
Evizeon~\cite{Hoque2017}, 
another natural language interface for visualization
specification, 
makes use of anaphoric references 
to fill in incomplete follow-on queries. 
For example, when a user says 
`Show me average price by neighborhood', 
then `by home type', the system interprets 
the anaphoric reference as continuing the 
context of the original utterance related 
to average price on the y-axis. 
Semantic ambiguity can often be composed 
of one or more lexical and syntactical ambiguities. 
For example, in Iris~\cite{Fast2018}, a dialog system for
data science, users can specify a vague, 
high-level query such as `Create a classifier', 
then Iris makes use of nested conversations 
to inquire about what type of classifiers and features to use for the model, 
to fill in the details of the structure and parameters required. 

% \agp{cite causality work~\cite{meliou2010causality,roy2014formal}}
% \agp{One concern is that we're talking about systems that support visualization specification, or dialog, but none of them talk about visualization search. We should clarify that these systems are only described because they have encountered similar challenges, but our challenges will be broader and harder to optimize.}

\techreport{
\begin{figure}[h!]
\centering
\includegraphics[width=\textwidth]{figures/ambiguity.pdf}
\caption{Examples of fuzzy, complex queries: a) Eviza~\cite{Setlur2016} uses ambiguity widgets to enable users to clarify vaguely-defined terms in their input query to resolve lexical ambiguity; b) DataPlay~\cite{Abouzied2012} allow users to toggle between `for all' and `at least one' quantifier to control for syntactical ambiguity; c) Iris~\cite{Fast2018} accepts a vague high-level task description and gather the additional information required through follow-up questions in a nested conversation.}
\label{fig:ambiguity}
\end{figure}
}

%they emit Since most visual query systems operate on top of a querying language, %during visual data exploration

%If a semantically vague query is not expressible through existing \vidaql, more research may be necessary for developing novel discovery modules in the \vida ecosystem. %, since the operations involved in the query may not be covered by the limited workflow combinations in the PVQS. 

% Another form of syntactic ambiguity is how  through followup query tweaking 
% nested queries and anaphoric references~\cite{Hoque2017}. Show me relationship between price and -----, then "Break down by -----".
% user intention, composition , Accounting for user interaction, mental models. More global objective taking into account user with the goal of dataset understanding rather than task completion. Need for a unified framework of inference to take all of these into account (e.g. natural language, etc)